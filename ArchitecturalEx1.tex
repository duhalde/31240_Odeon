%%%%%%%%%%%%%%%%    Alle dokumentets opsæ tninger
\documentclass[a4paper,11pt,titlepage]{article}
\usepackage[english, danish]{babel}          % Brug af danske ord
\usepackage[utf8x]{inputenc}		% æ ,ø,å [Latin1] under UNIX.
\usepackage[T1]{fontenc}            % Bruger en rigtig font til dansk output.
\usepackage{amsmath}                % Hvis man vil lave nogle frække matematikting
\usepackage{amssymb}                %  ---"---
\usepackage{mathabx}				% indeholder bl.a. \convolution
\usepackage{MnSymbol}
\usepackage[pdftex]{color,graphicx} % For at indsætte billeder
% \usepackage{pdfsync}
%\lstset{captionpos=b,float=*}
\usepackage{subfigure}  % lave figurer side om side
\usepackage{fullpage} % fikser siden så man har små sidemarginer.
\usepackage[version=3]{mhchem}
\usepackage[plainpages=false]{hyperref}
\hypersetup{colorlinks=true, linkcolor=blue, citecolor=red, urlcolor=black}
\usepackage[numbers, square]{natbib} % kan bruges til "Jones et al. (1990)" referencer
\bibpunct{[}{]}{,}{n}{,}{,~} % sørger for at \cite[rong,yuc] giver [1,2]
\usepackage{url}
\usepackage[isbn,issn,url]{dk-bib}
\usepackage{marginnote}
% \usepackage{mcode}
\usepackage{parskip}
\usepackage{tikz} 
\usepackage{pgfplots}
\pgfplotsset{compat=newest} 
\usetikzlibrary{external,plotmarks}

%\tikzexternalize[prefix=tikz/]
\newlength\figureheight 		% define length \figureheight
\newlength\figurewidth 			% define length \figurewidth

% \parindent=0pt % sletter den åndssvage indentering ved linieskift.
\newcommand{\eq}[1]{\begin{align*}#1\end{align*}}
\newcommand{\vnorm}[1]{\left|\!\left|#1\right|\!\right|}

%% Define a new 'leo' style for the package that will use a smaller font.
\makeatletter
\def\url@leostyle{%
  \@ifundefined{selectfont}{\def\UrlFont{\sf}}{\def\UrlFont{\small\ttfamily}}}
\makeatother
%% Now actually use the newly defined style.
\urlstyle{leo}

\graphicspath{pictures/}

% Start af dokumentet

\begin{document}	

\begin{titlepage}
\centering \parindent=0pt
\newcommand{\HRule}{\rule{\textwidth}{1mm}}
\vspace*{\stretch{1}} \HRule\\[1cm]\Huge\bfseries
31240\\Architectural Acoustics\\Exercise Report\\[0.7cm]
\large Ex. 0/1 Odeon Simulation\\[1cm]
\HRule\\[4cm]  \large by David Duhalde Rahbæk, s062050\\
Oliver Ackermann Lylloff, s082312\\
Mathias Immanuel Nielsen, s072101\\
Team 1\\
\vspace*{\stretch{2}} \normalsize %
\begin{flushleft}
Technical University of Denmark\\
Department of Acoustic Technology\\
Supervisor: Cheol-Ho Jeong\\
\today \end{flushleft}
\end{titlepage}

\newpage

\tableofcontents
\newpage
\section{Introduction} % (fold)
\label{sec:introduction}
The exercise concerns modelling of a room in the Odeon software. The exercise consist of two parts. The first is about familirization with the software, this is done by examining a selfmade model of an auditorium at DTU, made with Google SketchUp.
In the second part a provided model of the danish theater \O stre Gasv\ae rk is examined. This part is further divided into two sections; one concerning the exploration of the acoustics of the room, and one improving the acoustics.
% section introduction (end)


%\input{theory.tex}
\section{Exercise 0} % (fold)
\label{sec:exercise0}
The model of the auditorium was created in Google Sketchup. This is a simple modelling tool making use of drawing 2D-shapes and extruding them into the room. Using Google Sketcup was quite easy and most of the time was spent learning by watching the available introduction videos.\\
\subsection{Difficulties}
The main difficulty was with drawing the audience seating area. Drawing a skewered reactangle led to different solutions within the group. It would have been nice with general guidelines to the technique  used when drawing a room with inventory.\\
The use of layers could have been stressed especially since in apparently is the only way to group surfaces in Odeon. 

% section experimental_setup (end)

\section{Exercise 1} % (fold)
\label{sec:exercise1}
The aim of the second part of this exercise is to improve the reverberation time of the provided model of \O stre Gasv\ae rk. The task is to reduce the reverberation time to T $\approx 1.4$~s, by analyzing the reflections in Odeon and adding absorbing materials to critical surfaces.
\subsection{Locating the problem}
First we remove the proscenium wall and the reflector above the scene (or rather make them transparent by setting material = 0). Then we run Ray Tracing and get the reverberation time from Global estimate.

\begin{center}
\begin{tabular}{ccccccccc}
\hline
Band [Hz]   &       63    &   125   &    250    &   500  &    1000   &   2000   &   4000  & 8000 \\
T(30)  [s]   &   2,96   &   2,94  &    5,08   &   5,32   &   5,60  &    4,53  &    2,90  &    1,22 \\
\hline
\end{tabular}
\end{center}

This is quite bad considering the fact that this room is intended to be used for music and theatre. 

It is clear from the reverberation times that we need to put in a lot of absorption in order to get near the goal of T = 1.4 s. It is however not irrelevant where we put the absorbing material because some surfaces are more dominant reflectors than others. Luckily we can locate the origin of reflections as function of delay time in Odeon. Two groups of reflections cluster around 70ms and 230ms. The early reflections arrive primarily from the ceiling and the late from the ceiling (after the first reflection has hit the ceiling and bounced off the floor behind the audience). 

So the main concern for our improvements are on the ceiling and floors. 
\subsection{Improvements}
First of all, we model the room by adding audience to the tribunes. This `material' acts as a scatter and quite good absorber especially at high frequencies ($\alpha_{\text{low}}\approx 0.5$ and $\alpha_{\text{high}}\approx 0.8$).

However this is not enough to lower the reverberation time and the strong reflections are still causing an audible echo. In our improvements we want to add two types of absorbers. 
\begin{itemize}
\item Thin porous material suspended in front of hard wall. Generally porous materials are good absorbers at high frequencies but quite poor at low frequencies. By suspending the porous material from the hard wall, an air gap is created and the resulting absorption coefficient is almost constant over the frequency range, see figure 1.7 \cite[p. 10]{Rindel1981absorbers}).
\item Wooden panels in front of a cavity will generally give modest reverberation at low frequencies. By placing the wooden panels on the floor, a large surface area is covered in order to lower the reverberation time at low frequencies.
\end{itemize}
In addition we look specifically at the reflection with a time delay of 70 ms. Figure 2.4-2.9 in \cite[pp. 8-11]{Gade2003room} show the desired values of the relative level of the reflected signal in order to avoid echoes in music and speech. If the absorbers are not able to dampen the reflections at 70 ms it can be necessary to add e.g. diffusers or reflectors to dampen these early reflections.


\section{Discussion} % (fold)
\label{sec:discussion}
When listening to the binaural response through headphones, we observe a substantial improvement in the perceived speech intelligibility and sound quality, when listening to music in the theatre.\\
The decrease in reverberation time is seen in tables~\ref{tab:before}~and~\ref{tab:after}, where T(30) for 1~kHz has moved from 5.60~s to 1.22~s, and the SPL has balanced around 34-37~dB, in the 63-1000~Hz, against 35-46~dB before the acoustic treatment. And the perceived, i.e. A-weighted, sound pressure level, drops from 45.3~dB, to 39.2~dB.\\
Looking at the use of the hall, which is theatre, musicals, and concerts, the speech transmission index is important, and we managed to improve this from 0.40 (poor), to 0.74 (close to excellent).\\
The change in clarity is really massive, but might not be optimal for the purpose. The clarity values are extremely high, which indicates, that we were succesfull in damping the late reflections (Clarity is the ratio of early to late reflections). 

% section discussion (end)

\begin{table}[htdp]
\begin{center}\begin{tabular}{lcccccccc}
Band [Hz]   &       63    &   125   &    250    &   500  &    1000   &   2000   &   4000  & 8000 \\
\hline
T(30)  [s]   &   2.96   &   2.94  &    5.08   &   5.32   &   5.60  &    4.53  &    2.90  &    1.22 \\
EDT    [s]   &   3.40 &     3.38  &    5.15  &    5.41   &   5.58   &   4.61    &  3.28   &   2.46  \\
Ts     [ms] &  134  &     135   &    311    &   358   &    364  &     343 &      225  &     101    \\
SPL    [dB] &   35.1   &   35.2   &   42.2  &    46.0   &   40.3   &   31.6  &    23.1  &    11.1   \\
D(50)       &  0.53   &   0.51   &   0.30   &   0.23  &    0.23   &   0.18   &   0.28   &   0.49  \\
C(80)  [dB]  &   3.3     &  3.2   &   -1.6  &    -3.4  &    -3.4  &    -4.5  &    -2.4  &     2.2   \\
LF(80)    &      0.108   &  0.109    & 0.120   &  0.130  &   0.134 &    0.150 &    0.133   &  0.122
\end{tabular} 
\caption{Acoustic parameters before. SPL(A) = 45.3 dB and STI = 0.40.}
\end{center}
\label{tab:before}
\end{table}



\begin{table}[htdp]
\begin{center}\begin{tabular}{lcccccccc}
Band [Hz]   &       63    &   125   &    250    &   500  &    1000   &   2000   &   4000  & 8000 \\
\hline
T(30)  [s]   &   1.67    &  1.66  &    1.50     & 1.30   &   1.22   &   1.11   &   1.05  &    0.83  \\
EDT    [s]   &   1.47   &   1.46  &    0.90   &   0.73   &   0.68   &   0.67   &   0.73   &   0.63   \\
Ts     [ms] &  48   &     49   &     42  &      36   &     28   &     30   &     24   &     16        \\
SPL    [dB] &   34.0   &   34.1  &    37.6   &   39.8   &   33.9  &    24.2   &   17.3   &    7.6     \\
D(50)       &  0.74   &   0.72  &    0.74    &  0.74    &  0.79     & 0.77   &   0.80   &   0.84   \\
C(80)  [dB]  &   8.4   &    8.3   &    8.9   &    9.4   &   10.7   &   10.3   &   11.3  &    14.2     \\
LF(80)    &      0.183 &    0.183  &   0.151  &   0.107  &   0.090  &   0.100   &  0.061   &  0.056
\end{tabular} 
\caption{Acoustic parameters before. SPL(A) = 39.2 dB and STI = 0.74.}
\end{center}
\label{tab:after}
\end{table}



\newpage
\section{Conclusion} % (fold)
\label{sec:conclusion}

% \newpage
% \appendix
\newpage
% \bibliographystyle{is-unsrt} % prøv også: is-unsrt, alpha, plain
% \bibliography{skabelon-bib}
% %\nocite{*} % \nocite bruges til bøger som skal med i litteraturlisten
%             % men som ikke er refereret til. Altså baggrundsstof mv.
\addcontentsline{toc}{section}{References}\label{cha:bib}
\bibliographystyle{unsrtnat}
\bibliography{bib/bibliography.bib}

%\citationstyle{agsm}
%\nocite{*} %

% 
% \newpage
% \section{Litteraturliste}
% 
% %\bibliographystyle{unsrt}
% \begingroup
% \hypersetup{linkcolor=black}
% \printbibliography
% \endgroup

\newpage
\appendix

\end{document}
